\usepackage{graphicx}
\graphicspath{{./images/}}
\usepackage{amsmath,amsthm,amssymb}
\usepackage[numbers]{natbib}
\usepackage{xr-hyper}
\usepackage[colorlinks=true,linkcolor=blue,citecolor=blue,anchorcolor=blue]{hyperref}
\usepackage[capitalise]{cleveref}
% \usepackage{natbib}
\usepackage{color}
\usepackage{acro}
\usepackage{mathtools}
\usepackage{booktabs}
\usepackage{pdfpages}
\usepackage{lipsum}

\DeclareMathOperator*{\argmin}{arg\,min}

\newcommand{\grad}{\nabla}

\newcommand{\set}[1]{\{ #1 \}}
\newcommand{\Set}[1]{\left\{ #1 \right\}}

\newcommand{\mat}[1]{\bm{#1}}
\newcommand{\reals}{\mathbb{R}}
\newcommand{\wrt}{\textrm{d}}
\newcommand{\abs}[1]{|#1|}
\newcommand{\Norm}[1]{\left\|#1\right\|}
\newcommand{\norm}[1]{\|#1\|}
\newcommand{\bignorm}[1]{\big\|#1\big\|}
\newcommand{\inner}[1]{\langle #1 \rangle}
\newcommand{\Inner}[1]{\left\langle #1 \right\rangle}
\newcommand{\quark}{\setbox0\hbox{$x$}\hbox to\wd0{\hss$\cdot$\hss}}

\newcommand{\defeq}{\coloneqq}
\newcommand{\op}{\mathsf{op}}

\newcommand\numberthis{\addtocounter{equation}{1}\tag{\theequation}}


\newcommand{\email}[1]{\protect\href{mailto:#1}{#1}}

\newcommand{\todo}[1]{{\color{red}{#1}}}
\newcommand\funding[1]{\protect\\ \hspace*{1.8em}{\bfseries Funding:} #1}
\newcommand{\mylozenge}{$\mathbin{\blacklozenge}$}
\newcommand{\R}{\mathbb{R}}
\newcommand{\E}{\mathbb{E}}

\renewcommand{\qedsymbol}{$\blacksquare$} % remove if want empty square and end of proofs

\DeclareMathOperator{\tr}{tr}
\DeclareMathOperator{\var}{var}

\newtheorem{theorem}{Theorem}

\newtheorem{proposition}{Proposition}

\theoremstyle{definition}

\newtheorem{assumption}{Assumption}
\crefname{assumption}{Assumption}{Assumptions}

\theoremstyle{remark}
\newtheorem{remark}{\textbf{Remark}}  % if don't want bold face for remark remove textbf
\crefname{remark}{Remark}{Remarks}

% \theoremstyle{remark}
% \newtheorem*{remarks}{\textbf{Remarks}}
% \crefname{remarks}{Remarks}{Remarks}

\theoremstyle{definition}
\newtheorem{lemma}{Lemma}

\DeclareAcronym{statfem}{short=StatFEM, long=statistical finite element method}
\DeclareAcronym{fem}{short=FEM, long=Finite Element Method}
\DeclareAcronym{pde}{short=PDE, long=partial differential equation}
\DeclareAcronym{ode}{short=ODE, long=ordinary differential equation}
\DeclareAcronym{rkhs}{short=RKHS, long=reproducing kernel Hilbert space}
\DeclareAcronym{gp}{short=GP, long=Gaussian process,short-plural=s,long-plural=es}
\DeclareAcronym{bvp}{short=BVP, long=boundary value problem}
\DeclareAcronym{femesh}{short=FE mesh, long=finite element mesh,short-plural=es,long-plural=es}
\DeclareAcronym{dof}{short=DOF, long=degree of freedom,long-plural-form=degrees of freedom}
\DeclareAcronym{fe}{short=FE, long=Finite Element}
\DeclareAcronym{pn}{short=PN, long=Probabilistic Numerics}
\DeclareAcronym{pnm}{short=PNM, long=Probabilistic Numerical Method}
\DeclareAcronym{ivp}{short=IVP, long=initial value problem}
\DeclareAcronym{gmrf}{short=GMRF, long=Gaussian Markov Random Field}

% \title{Theoretical Guarantees for the Statistical Finite Element Method}

\author{Yanni Papandreou\thanks{Imperial College London (\email{john.papandreou18@imperial.ac.uk})} \and Jon Cockayne \thanks{University of Southampton (\email{jon.cockayne@soton.ac.uk})} \and Andrew B. Duncan\thanks{Imperial College London and The Alan Turing Institute (\email{a.duncan@imperial.ac.uk})}}
